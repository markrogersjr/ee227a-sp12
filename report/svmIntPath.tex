\documentclass[11pt]{article}
\usepackage{amssymb,amsmath,amsthm}
\newtheorem*{theorem}{Theorem}
\newtheorem{lamma}{Lemma}[section]
\theoremstyle{definition}
\newtheorem*{definition}{Definition}  
%\setlength{\oddsidemargin}{0in} 
%\setlength{\textwidth}{6.5in}
%\setlength{\headheight}{0in}
%\setlength{\headsep}{0in}
%\setlength{\topmargin}{0in}
%\setlength{\textheight}{9in}
%\pagestyle{empty}
\usepackage{graphicx}
\usepackage{enumerate}
%\documentclass[11pt]{article}
\pagestyle{empty}
\oddsidemargin=0in
\topmargin=-.5in
\textwidth=6.5in
\textheight=8.6in


\begin{document}

\title{The Entire Regularization Path for the Interval Support Vector Machine}
\maketitle
\section{Problem formulation}
Suppose $n$ data points are given: $(x_i,y_i)$ for $i=1,...,n$. Let $\mathcal I_+ = \{i: y_i=+1\}$ and $\mathcal I_- = \{i: y_i=-1\}$ be the set of positive class and negative class respectively. Also let $n_+=|\mathcal I_+|$ and $n_-=|\mathcal I_-|$ be the number of data points in these two classes. In this section, we will derive an efficient algorithm to compute the entire regularization path for the following interval svm problem:
\begin{equation}
\label{eq:svmInit}
\min_{\beta}\sum_{i=1}^n(1-y_i(\beta_0+\beta^Tx_i)+\rho\sigma_i^T|\beta|)_++\frac{\lambda}{2} ||\beta||_2^2.
\end{equation}
\eqref{eq:q}
(\ref{eq:q})
Introducing slack variables to get rid of the hinge loss function and absolute values, the problem is equivalent to:
\begin{align}
\label{eq:primal}
\min & \sum_{i=1}^n\xi_i + \lambda\frac{\beta^T\beta}{2}, \\
\text{subject to } & \xi_i \geq 1 - y_i(\beta_0+\beta^Tx_i)+\rho \sigma_i^Tt,\nonumber \\
                   & \xi_i \geq 0, \text{ for }i=1,2,...,n; \nonumber\\
                   & -t_j\leq \beta_j \leq t_j,\nonumber\\
                   & t_i \geq 0, \text{ for } j=1,2,...,p. \nonumber                      
\end{align}
Let the above problem be the primal problem. The Lagragian for this problem is
\[
\begin{array}{rl}
L(\xi,\beta_0,\beta,t,\alpha,\gamma,\nu,\mu,c)  = & \sum_{i=1}^n \xi_i + \lambda \frac{\beta^T\beta}{2} + \sum_{i=1}^n\alpha_i(1 - y_i(\beta_0+\beta^Tx_i)+\rho \sigma_i^Tt-\xi_i) \\
& - \sum_{j=1}^p\mu_j(t_j-\beta_j) - \sum_{j=1}^p\nu_j(t_j+\beta_j) - \sum_{i=1}^n\gamma_i\xi_i - \sum_{i=1}^nc_jt_j. \\
= & \sum_{i=1}^n(1-\alpha_i-\gamma_i)\xi_i + \lambda\frac{\beta^T\beta}{2} - (\sum_{i=1}^n \alpha_iy_ix_i - (\mu-\nu))^T\beta \\
  & - \sum_{i=1}^n\alpha_iy_i\beta_0 + \sum_{j=1}^p(\rho\alpha_i\sigma_{ji} - \mu_j-\nu_j-c_j)t_j.
\end{array}
\]
Minimizing with respect to the primal variables $(\xi,\beta_0,\beta,t)$ we derive the dual problem as follows:

\begin{align}
\label{eq:dual}
\max_{\alpha,\mu,\nu} & \sum_{i=1}^n \alpha_i - \frac{1}{2\lambda}||\sum_{i=1}^n \alpha_iy_i x_i - (\mu-\nu)||_2^2, \\
\text{subject to } & \sum_{i=1}^n \alpha_i y_i = 0, \ \rho \sum_{i=1}^n \alpha_i \sigma_{i} \geq \mu + \nu, \nonumber \\
& \alpha \in [0,1], \ \mu\geq0, \ \nu\geq0. \nonumber
\end{align}

Remarks: if the precision matrix is zero, i.e. $\Sigma = 0$, then second dual constraint implies $\mu+\nu\leq0$. Furthermore by the nonnegativity constraints we have $\mu=0$ and $\nu=0$. So the dual problem becomes
\[
\begin{array}{rl}
\max_{\alpha} & \sum_{i=1}^n \alpha_i - \frac{1}{2\lambda}||\sum_{i=1}^n \alpha_i y_i x_i||_2^2, \\
\text{subject to } & \sum_{i=1}^n \alpha_i y_i = 0, \ \alpha \in [0,1],
\end{array}
\]
which is exactly the dual to the standard SVM (see, for example, (12.13) of EML).\\

One can easily determine the KKT conditions of the Interval SVM problem:\\
(1) Primal stationarity:
\begin{align}
\frac{\partial}{\partial \beta}: & \ \beta = \frac{1}{\lambda}(\sum_{i=1}^n \alpha_i y_i x_i + \nu-\mu) \\ 
\frac{\partial}{\partial \beta_0}: & \ \sum_{i=1}^n \alpha_i y_i = 0\\
\frac{\partial}{\partial \xi}: & 1-\alpha - \gamma =0.
\end{align}
(2) Complementary slackness:
\begin{align}
\alpha_i(1 - y_i(\beta_0+\beta^Tx_i)+\rho \sigma_i^Tt-\xi_i) & = 0 \\
\gamma_i\xi_i & = 0 \\
\mu_j(t_j-\beta_j) & = 0 \\
\nu_j(t_j+\beta_j) & = 0 \\
c_jt_j & = 0 \\
\rho \sum_{i=1}^n \alpha_i \sigma_{ji} - (\mu_j +\nu_j+c_j) & =0 \\
\end{align}
(3) Primal feasibility and; (4) Dual feasibility.

From the optimality conditions, it is easy to observe the following statements hold:\\
(i) If $\xi_i>0$, then $\gamma_i = 0$ and so $\alpha_i =1$. Therefore by equation (7) we have $y_i(\beta_0+\beta^Tx_i)<1+\rho \sigma_i^Tt$, which indicates that the data point $i$ is on the left of the elbow of the hinge loss. \\
(ii) If $y_i(\beta_0+\beta^Tx_i)> 1+\rho \sigma_i^Tt$, that is, when the $i$-th data point is on the right of the elbow, then $\xi=0$ and by equation (6) and (8) $\alpha_i =0$. \\
(iii) If $y_i(\beta_0+\beta^Tx_i) =  1+\rho \sigma_i^Tt$, that data point is on the elbow. In this case we can only know $\alpha_i\in[0,1]$.\\
%(iv) if $t_j>0$, then $c_j=0$. By equation (XX), $\rho\sum_{i=1}^n \alpha_i \sigma_{ji} = \mu_j+\nu_j$
(iv) It is obvious that at optimal, we have $t = |\beta|$.\\
Now for convenience let $f(x) = \beta_0+\beta^Tx$ and define the following index sets for the data points:
\[\mathcal E = \{i: y_if(x_i) =  1+\rho \sigma_i^T|\beta|, 0\leq\alpha_i\leq1\}, \ \mathcal E \text{ for Elbow},\]
\[\mathcal L = \{i: y_if(x_i) <  1+\rho \sigma_i^T|\beta|, \alpha_i = 1\}, \ \mathcal L \text{ for Left of the elbow},\]
\[\mathcal R = \{i: y_if(x_i) >  1+\rho \sigma_i^T|\beta|,\alpha_i =0 \}, \ \mathcal R \text{ for Right of the elbow}.\]
Note that the above sets are similar to those defined in \cite{key-2}. In addition, since the absolute value of $\beta$ is involved, we also need to keep track of the following index sets for the parameter vector $\beta$:
\[\mathcal V_+ = \{j: \beta_j>0, \nu_j=0\},\]
\[\mathcal V_- = \{j: \beta_j<0,\mu_j=0\},\]
\[\mathcal Z = \{j: \beta_j=0,\sum_{i=1}^n \alpha_i \sigma_{ji} \geq \mu_j + \nu_j\}.\]
\section{Initialization}
We will determine the initial state of the parameters and the sets defined above. Without loss of generality, we assume that $n_+\geq n_->0$. We start with $\lambda=+\infty$ or equivalently $C=0$. In this case, it is easy to see that $\beta=0$. 
\begin{lamma}
Given $n_+\geq n_->0$ and $\beta=0$, the optimal value for the initial value of $\beta_0$ is $\beta_0 = 1$ and the loss is $\sum_{i=1}^n \xi_i = 2n_-$. If $n_+>n_-$, then $\beta_0 = 1$.\\
\end{lamma}
{\textbf Proof.} With $\beta=0$, the primal problem becomes:
\[
\begin{array}{rl}
\min & \sum_{i=1}^n \xi_i, \\
\text{subject to } & \xi_i \geq 0, \ \xi_i \geq 1- y_i \beta_0.
\end{array}
\]
Hence for $i\in \mathcal I_+$, $\xi_i \geq 1-\beta_0$ and for $i\in \mathcal I_-$, $\xi_i \geq 1+\beta_0$. At optimal, the equal sign must hold since we are minimizing the sum $\sum_{i=1}^n \xi_i$. Also, $\xi_i$'s are nonnegative, therefore we have $1-\beta_0\geq0$ and $1+\beta_0\geq0$ and so $\beta_0\in [-1,1]$. Furthermore, the objective function can be written in terms of $n_+$ and $n_-$:
\[
\sum_{i=1}^n \xi_i = n_+(1-\beta_0)+ n_-(1+\beta_0) = (n_--n_+)\beta_0 + (n_++n_-), 
\]
which is a linear function in $\beta_0$. If $n_-=n_+$, then the above function is simply a constant $2n_-$ and so $\beta_0$ can be arbitrary chosen in $[-1,1]$; on the other hand if $n_+>n_-$, the linear function is decreasing in $\beta_0$ and so to minimize the sum, we can pick $\beta_0$ on the right end point of $[-1,1]$. So $\beta_0 = 1$ if $n_+>n_-$. \\

From now on, we can safely assume that $n_+>n_-$, in which case the initial state is $\beta = 0$, $\beta_0=1$, $\xi_i = 0$ for $i\in \mathcal I_+$ and $\xi_i = 2$ for $i\in \mathcal I_-$. Thus by optimality conditions, for any $i\in \mathcal I_-$, $\gamma_i =0$ and so $\alpha_i=1$. In addition, since $0=\sum_{i=1}^n \alpha_i y_i = \sum_{i\in \mathcal I_+} \alpha_i - \sum_{i\in \mathcal I_-} \alpha_i$, we have $\sum_{i\in \mathcal I_+} \alpha_i = \sum_{i\in \mathcal I_-} \alpha_i = n_-$. The following lamma determines the initial value for the dual valuables:
\begin{lamma} Let $\tilde{\beta} = \sum_{i=1}^n \alpha_i y_i x_i + \nu - \mu$, 
\[
\begin{array}{rl}
(\alpha^*,\mu^*,\nu^*) = & \arg\min ||\tilde{\beta}||_2^2, \\
& \text{subject to: } 0\leq \alpha_i\leq 1,  \forall i \in \mathcal I_+; \alpha_i = 1,  \forall i \in \mathcal I_-; \\
& \sum_{i \in \mathcal I_+}\alpha_i = n_-, \rho\Sigma\alpha \geq \mu+\nu, \mu\geq0, \nu\geq0. \\
\end{array}
\]
 and $c^* = \rho\Sigma\alpha^* - \mu^*-\nu^*$. Then for some $\lambda_0$ we have for all $\lambda>\lambda_0$, $(\alpha(\lambda),\mu(\lambda),\nu(\lambda),c(\lambda)) = (\alpha^*, \mu^*,\nu^*,c^*)$.
\end{lamma}  
\textbf{Proof.} See \cite{key-2}.\\
Now let $\beta^* = \sum_{i=1}^n \alpha_i^* y_i x_i + \nu^* - \mu^*$ be the fixed coefficient direction corresponding to the initial $(\alpha^*,\mu^*,\nu^*)$. By lemma (XX) and the optimality conditions, we have for all $\lambda>\lambda_0$:
\[\beta(\lambda) = \frac{\sum_{i=1}^n \alpha_i^* y_i x_i + \nu^* - \mu^*}{\lambda}.\]
We now need to determine the point $\lambda_0$ that the dual variables $(\alpha,\mu,\nu,c)$ start to change. At the very beginning when $\lambda=+\infty$, there are two possible scenarios:
\begin{itemize}
\item There exist at least two data points in $\mathcal I_+$ with $0<\alpha_i^*<1$. Notice that there cannot be only one such element due to the integer constraint $\sum_{i\in \mathcal I_+} \alpha_i^* = n_-$. 
\item For all $i\in \mathcal I_+$, $\alpha_i^*$ is either 0 or 1.
\end{itemize}
For the first scenario, arbitrary pick an $i_+ \in \mathcal I_+$ such that $0<\alpha_{i_+}^*<1$. Since any of these points will stay in the elbow until a point in $\mathcal I_-$ enters the elbow, we can consider the element in $\mathcal I_-$ that first reaches its margin, i.e. $i_- = \arg \min_{i \in \mathcal I_-} x_i^T\beta^* + \rho \sigma_i^T|\beta^*|$. Therefore at $\lambda=\lambda_0$, as both $i_+$ and $i_-$ are at their respective margin, by the definition of elbow, we have the following system of equations:
\[\beta_0+\frac{1}{\lambda}x_{i_+}^T\beta^* =  1+\frac{1}{\lambda}\rho \sigma_{i_+}^T|\beta^*|,\]
\[\beta_0+\frac{1}{\lambda}x_{i_-}^T\beta^* =  -1-\frac{1}{\lambda}\rho \sigma_{i_-}^T|\beta^*|.\]
Solving for $\lambda$ and $\beta_0$ yields:
\[\lambda_0 = \frac{1}{2}(\beta^{*T}(x_{i_+}-x_{i_-})-\rho|\beta^*|^T(\sigma_{i_+}+\sigma_{i_-})),\]
\[\beta_0 = \frac{-\beta^{*T}(x_{i_+}+x_{i_-})+\rho|\beta^*|^T(\sigma_{i_+}-\sigma_{i_-})}{\beta^{*T}(x_{i_+}-x_{i_-})-\rho|\beta^*|^T(\sigma_{i_+}+\sigma_{i_-})}.\]
For the second scenario, for the initial parameter to change, a point in $\mathcal I_-$ and a point in $\mathcal I_+$ with $\alpha_i^* = 1$ must reach the margin simultaneously. Therefore we can let $i_+ = \arg \max_{i\in \mathcal I_+, \alpha_i=1} x_i^T\beta^* - \rho \sigma_i^T|\beta^*|$ and obtain $\lambda_0$ and $\beta_0$ by solving the above equations.\\

Remarks: in \cite{key-2}, Hastie et.al. split the initialization step into two cases depending on whether $n_+=n_-$ or $n_+>n_-$. In their case $n_+=n_-$, the initial state can be obtained efficiently without solving the quadratic programming problem. This does not seem to be the case in our problem set-up, as it also involves finding the initial values of the extra dual variables $\mu$ and $\nu$. Therefore solving a quardratic problem for starting up the path algorithm cannot be avoided.\\

\section{The Path}
Our algorithm keeps track of the following events:
\begin{enumerate}
\item[1.] One or more points from $\mathcal L$ have just entered $\mathcal E$;
\item[2.] One or more points from $\mathcal R$ have just reentered $\mathcal E$;
\item[3.] One or more points in $\mathcal E$ have just left the set, to join either $\mathcal R$ or $\mathcal L$.
\end{enumerate}
\noindent The above are events that are considered by \cite{key-2} for deriving a path algorithm for standard SVM. Since our problem involves also the $l_1$ norm of the parameter $\beta$, we also need to keep track of the following event:
\begin{enumerate}
\item[4.] One or more indices of the parameter $\beta$ enter $\mathcal Z$, that is,  $\beta_j$ becomes zero, initially being either positive or negative;
\item[5.] One or more indices of the parameter $\beta$ enter $\mathcal V_+$ from $\mathcal Z$;
\item[6.]  One or more indices of the parameter $\beta$ enter $\mathcal V_-$ from $\mathcal Z$.
\end{enumerate}
\subsection{Piecewise Linearity between Events}
By continuity, all of the index sets defined will stay the same until the next event occurs. Index by the superscript $l$ the sets immediately after the $l$-th event occurs. Likewise, index all the parameters in the same manner, and let $f^l$ be the classify function at this point. Also, for notation consistency define $\alpha_0 = \lambda \beta_0$ and so $\alpha_0^l = \lambda^l \beta_0^l$. Consider for $\lambda^l > \lambda > \lambda^{l+1}$, we have:
\begin{align}
\label{eq:classifier}
f(x) & = [f(x) - \frac{\lambda^l}{\lambda}f^l(x)] + \frac{\lambda^l}{\lambda}f^l(x) \nonumber \\
& = \frac{1}{\lambda}[x^T(\sum_{i=1}^n\alpha_iy_ix_i+\nu-\mu)+\alpha_0 - x^T(\sum_{i=1}^n\alpha_i^ly_ix_i+\nu^l-\mu^l)-\alpha_0^l+\lambda^l f^l(x)] \nonumber \\
& = \frac{1}{\lambda}[x^T\sum_{i\in\mathcal E^l}(\alpha_i-\alpha_i^l)y_ix_i+x^T(\nu-\nu^l)-x^T(\mu-\mu^l)+(\alpha_0-\alpha_0^l) +\lambda^l f^l(x)],
\end{align}
where the last equality follows from the fact that $\{1,...,n\}=\mathcal L\cup\mathcal E\cup \mathcal R$, and $\alpha_i = 1$ on $\mathcal L$; $\alpha_i=0$ on $\mathcal R$. The value of $\alpha_i$ for $i$ not in the elbow remains unchanged between events. Since each data point in $\mathcal E^l$ is to stay in $\mathcal E$ for $\lambda\in (\lambda^{l+1},\lambda^{l})$, we have:
\[y_jf(x_j) = 1+ \rho\sigma_j^T|\beta|.\]
Plug in equation \eqref{eq:classifier} for $f(x_j)$, the above formula becomes:
\begin{align}
\label{eq:elbow}
\frac{1}{\lambda}[\sum_{i\in\mathcal E^l}(\alpha_i-\alpha_i^l)y_jy_ix_j^Tx_i+y_jx_j^T(\nu-\mu-(\nu^l-\mu^l))+y_j(\alpha_0-\alpha_0^l) +\lambda^l (1+\rho \sigma_j^T|\beta^l|)] = 1+\rho\sigma_j^T|\beta|.
\end{align}
We can expression the parameter $\beta_i$ in terms of $\alpha$. For any $i\in\mathcal V_+$, since $\nu_i=0$, $\mu_i = \rho\sum_{k=1}^n \alpha_k\sigma_{ik}$. Plug all these parameters into the formula for $\beta_i$, we have
\[
\beta_i = \frac{1}{\lambda}\sum_{k=1}^n(\alpha_iy_kx_{ik}-\mu_i) =\frac{1}{\lambda}\sum_{k=1}^n\alpha_k(y_kx_{ik}-\rho\sigma_{ik}) = \frac{1}{\lambda}\sum_{k=1}^n\alpha_kw_{ik}.\]
Similarly, for $i\in\mathcal V_-$, since $\mu_i=0$, $\nu_i = \rho\sum_{k=1}^n \alpha_k\sigma_{ik}$ and so
\[
\beta_i = \frac{1}{\lambda}\sum_{k=1}^n(\alpha_iy_kx_{ik}+\nu_i) =\frac{1}{\lambda}\sum_{k=1}^n\alpha_k(y_kx_{ik}+\rho\sigma_{ik})=\frac{1}{\lambda}\sum_{k=1}^n\alpha_kz_{ik}.\]
Let $h_k = \sum_{i\in\mathcal V_+}\sigma_{ij}w_{ik} - \sum_{i\in\mathcal V_-}\sigma_{ij}z_{ik}$. By splitting $\beta$ into positive and negative parts, we can get rid of the absolute value in $\sigma_j^T|\beta|$:
\[
\begin{array}{rl}
\sigma_j^T|\beta| &= \sum_{i\in\mathcal V_+}\sigma_{ij}\beta_{i} - \sum_{i\in\mathcal V_-}\sigma_{ij}\beta_{i} \\
&=\frac{1}{\lambda}\sum_{k=1}^n\alpha_k(\sum_{i\in\mathcal V_+}\sigma_{ij}w_{ik} - \sum_{i\in\mathcal V_-}\sigma_{ij}z_{ik})\\
&=\frac{1}{\lambda}\sum_{k=1}^n\alpha_kh_k
\end{array}
\] 
Next, let us look at the term $x_j^T(\nu-\mu)$:
\[
\begin{array}{rl}
x_j^T(\nu-\mu) & = \sum_{i\in\mathcal V_-}x_{ij}\nu_i - \sum_{i\in\mathcal V_+}x_{ij}\mu_i + \sum_{i\in Z}x_{ij}(\nu_i-\mu_i)\\
& = \sum_{i\in\mathcal V_-}x_{ij}(\rho\sum_{k=1}^n\alpha_k\sigma_{ik}) - \sum_{i\in\mathcal V_+}x_{ij}(\rho\sum_{k=1}^n\alpha_k\sigma_{ik}) + \sum_{i\in Z}x_{ij}(\nu_i-\mu_i)
\end{array}
\]
Notice that for $i\in Z$, $\beta_i=0$ and so $\sum_{k=1}^n\alpha_ky_kx_{ik}+\nu_i-\mu_i=0$. Therefore $\nu_i-\mu_i = -\sum_{k=1}^n\alpha_ky_kx_{ik}$. Hence we have:
\[
x_j^T(\nu-\mu) = \sum_{k=1}^n\alpha_k(\rho\sum_{i\in\mathcal V_-}x_{ij}\sigma_{ik}-\rho\sum_{i\in\mathcal V_+}x_{ij}\sigma_{ik}+\sum_{i\in Z}y_kx_{ik}) = \sum_{k=1}^n\alpha_k g_{jk},
\]
where $g_{jk}=\rho\sum_{i\in\mathcal V_-}x_{ij}\sigma_{ik}-\rho\sum_{i\in\mathcal V_+}x_{ij}\sigma_{ik}+\sum_{i\in Z}y_kx_{ik}$. Now, let $\delta_k = \alpha_k^l-\alpha_k$. After some rearrangements, equation \eqref{eq:elbow} becomes:
\[
\sum_{k\in\mathcal E}\delta_k(y_jy_kx_j^Tx_k + y_j g_{jk} - \rho h_{jk}) + y_j\delta_0 = \lambda^l-\lambda,
\]
for all $j\in\mathcal E$. Thus by constructing a matrix $\mathbf{K}$ with entry $K_{jk} = y_jy_kx_j^Tx_k + y_j g_{jk} - \rho h_{jk}$, the above system of equations can be represented by the following matrix form:
\[
\mathbf{K}\boldsymbol{\delta} + \delta_0 \mathbf{y}_l = (\lambda^l-\lambda)\textbf{1},
\]
where $\mathbf{y}_l$ is a vector of length $m$ whose entries are $y_i$'s for $i\in \mathcal E$. In addition, since $\sum_{i=1}^n \alpha_iy_i = 0$, we have $\sum_{k\in\mathcal E}y_k\delta_k=0$ and so
\[\mathbf{y}_l^T\boldsymbol{\delta}=0.\]
So all together we have $m+1$ unknown variables and $m+1$ linear equations. Now let:
\[
\mathbf{A}_l = 
\left( \begin{array}{cc}
0 & \mathbf{y}_l^T \\
\mathbf{y}_l & \mathbf{K}_l 
\end{array}\right), \
\boldsymbol{\delta}^a = 
\left( \begin{array}{c}
\delta_0 \\
\boldsymbol{\delta}
\end{array}
\right), \ 
\textbf{1}^a = 
\left( \begin{array}{c}
0 \\
\textbf{1}
\end{array}
\right). 
\]
So the linear system can be written as
\[
\mathbf{A}_l\boldsymbol{\delta}^a = (\lambda_l-\lambda)\textbf{1}^a.
\]
Let $\textbf{b}^a = \mathbf{A}_l^{-1}\textbf{1}^a$, then for $k=0$ or $k \in \mathcal E_l$, we have
\begin{align}
\label{eq:alpha}
\alpha_k = \alpha_k^l - (\lambda^l-\lambda)b_k,
\end{align}
which shows linearity of $\alpha$ between two events, i.e. $\lambda^{l+1}<\lambda<\lambda^{l}$. To evaluate the function value of $f$ at the data point $x_j$, we have
\begin{align}
\label{eq:fxj}
f(x_j) = \frac{\lambda^l}{\lambda}(f^l(x_j)-h^l(x_j))+h^l(x_j),
\end{align}
where 
\[
h^l(x) = \sum_{k\in\mathcal E_l}b_j(y_kx_j^Tx_k+g_{jk})+b_0.
\]
\subsection{Finding the Next Event}
The linear (or inverse linear) path established above continues until one of the following events occur:
\begin{enumerate}
\item One of the points (if any) on the elbow $\mathcal E_l$ is about to enter either $\mathcal R$ or $\mathcal L$. For the first case $\alpha_i = 0$, by solving equation \eqref{eq:alpha} we can obtain a candidate breakpoint $\lambda$ for the $j$-th data point:
\[
\lambda = \frac{\lambda^lb_j-\alpha_j^l}{b_j}.
\]
Similar, in the case where the point is about to enter $\mathcal L$ we have
\[
\lambda = \frac{\lambda^lb_j-\alpha_j^l+1}{b_j}.
\]
Compute the above candidate break points for all data points that are currently on the elbow (so that we have $2m$ such candidates). Take the greatest such $\lambda$ that is smaller than $\lambda^l$, denote that quantity by $\lambda_a$.
\item One of the points $j$ in either $\mathcal L^l$ or $\mathcal R^l$ enters the elbow, that is $y_jf(x_j) = 1+\rho\sigma_j^T|\beta|$. By equation \eqref{eq:fxj}, we have
\[
\frac{\lambda^l}{\lambda}(f^l(x_j)-h^l(x_j))y_j + h^l(x_j)y_j = 1 + \frac{1}{\lambda}\sum_{i=1}^n\alpha_kh_k.
\]   
The right-hand-side of the above equation can be rewritten as
\[
1+\frac{1}{\lambda}\sum_{k\in\mathcal L}h_k + \frac{1}{\lambda}\sum_{k\in\mathcal E}(\alpha_k^l + (\lambda-\lambda^l)b_k)h_k.
\]
So a break point candidate is
\[
\lambda = \frac{\lambda^l(f^l(x_j)-h^l(x_{jk}))-y_j\sum_{k\in\mathcal L}h_{jk} - y_j\sum_{k\in\mathcal E}\alpha_k^lh_{jk} + \lambda^l\sum_{k\in\mathcal E}b_k h_{jk}}{y_j-h^l(x_j)+y_j\sum_{k\in\mathcal E}b_kh_{jk}}.
\]
\item A nonzero component of $\beta$ becomes zero. This means for each $i\in\mathcal V_+$, by equation  we solve $\sum_{k=1}^n\alpha_kw_{ik}=0$ and obtain a break point candidate
\[
\lambda = \frac{\lambda^l\sum_{k\in\mathcal E} b_kw_{ik}-\sum_{k\in\mathcal E}\alpha_k^l w_{ik}-\sum_{k\in\mathcal L} w_{ik}}{\sum_{k\in\mathcal E} b_kw_{ik}}.
\]
For $i\in\mathcal V_-$, simply replace $w_{ik}$ by $z_{ik}$.

\item A zero component of $\beta$ becomes nonzero. For a component $i\in Z$ to become positive, we require $c_i = 0$ and $\nu_i =0$. By equation (XX), $\mu_i = \sum_{k=1}^n\alpha_ky_kx_{ik}$, together with the optimality condition (XX), we need to solve $\rho\sum_{k=1}^n\alpha_k\sigma_{ik}=\sum_{k=1}^n\alpha_ky_kx_{ik}$. Same analysis for $i\in Z$ to become negative. Hence we have
\[
\lambda = \frac{\lambda^l\sum_{k\in\mathcal E} b_kw_{ik}-\sum_{k\in\mathcal E}\alpha_k^l w_{ik}-\sum_{k\in\mathcal L} w_{ik}}{\sum_{k\in\mathcal E} b_kw_{ik}}.
\]
  



\end{enumerate}
\section{Termination}
\section{Computation Complexity}
\section{Extension to Kernal} (You need to transform the uncertainty in the original data space into the kernal space)


\begin{thebibliography}{References}

\bibitem{key-1}Saharon Rosset, Ji Zhu. (2007)

\bibitem{key-2}T Hastie, S Rosset, R Tibshirani\ldots{} - The Journal
of Machine \ldots{}, 2004 - dl.acm.org

\bibitem{key-3}L El Ghaoui, G Lanckeriet, G Natsoulis. (2003) 

\bibitem{key-4} Element of machine learning
\end{thebibliography}


\end{document}