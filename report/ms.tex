\documentclass[11pt]{article}
\usepackage{amsmath,textcomp,amssymb,geometry,graphicx}
\newcommand{\argmin}[1]{\underset{#1}{\operatorname{argmin}}\text{ }}
\begin{document}
\title{Fast, Robust Classification with Applications to Periodic Variable Stars}
\author{Siqi Wu, Mark Rogers, and James Long}
\maketitle


\section{Introduction}

\section{The Data}
\subsection{Background on Periodic Variables}
Modern astronomical surveys observe millions of light sources (stars, galaxies, asteroids) over the course of a mission lasting a few years. Periodic variables, sources which vary periodically in brightness over time, are some of the most interesting. In the 1920's, periodic variables were crucial in Edwin Hubble's discovery the existence of galaxies \cite{berendzen1971hubble}. More recently, periodic variables have played an important role in determining expansion of the universe \cite{freedman2010hubble}.

Periodic variables may be divided into a few dozen classes based on physical properties of light sources. Separating the sources into classes is a critical step in turning raw astronomical observations into scientific knowledge. The size of modern data sets requires that much of this work be automated by machine learning and statistical classifiers. 
\begin{figure}[h]
  \begin{center}
    \begin{includegraphics}[scale=.5]{2000.pdf}
      \caption{Light curve of a Classical Cepheid type variable star. The brightness of the star is on the y-axis. For the top plot the brightness is plotted against time. This time series is periodic. Using fourier methods we can estimate a period of 4.51 days. We can convert the times from the top plot into phase ((time modulo period) / period)). Brightness versus phase is plotted in the bottom plot. Here we can observe the structure in the time series, which is usually similar for stars of the same class but different for stars of different classes.\label{fig:cepheid}}
    \end{includegraphics}
  \end{center}
\end{figure}

Figure \ref{fig:cepheid} displays the light curve (i.e. time series) of a periodic variable belonging to the class Classical Cepheid. The points in the top plot represent flux measurements in magnitudes (i.e. brightness of the source) made by the telescope at particular times. The 0 point on the time axis is arbitrary. Using fourier methods, one can estimate a period using these measurements.  The lower plot of Figure \ref{fig:cepheid} displays the flux measurements of the same object. However here the x-axis is phase of each time measurement, computed using the estimated period of 4.51 days. Here we can observe the structure of the periodic variation. This is known as the \textit{folded light curve}.

Figure \ref{fig:mira} displays an example of a Mira light curve. From the y-axis we can see that this source has higher amplitude than the Classical Cepheid (this is typical of the Mira class) and more sinusoidal variation (also typical). Note that the fourier methods appear to have estimated an incorrect period for this source. The true period appears to be around 161 days, half of the estimate.
\begin{figure}[h]
  \begin{center}
    \begin{includegraphics}[scale=.5]{204.pdf}
      \caption{Light curve of a Mira type variable star.\label{fig:mira}}
    \end{includegraphics}
  \end{center}
\end{figure}
\subsection{Classification Methodology}
There has been a lot of recent progress in the astronomy literature towards developing highly accurate classifiers for periodic variables \cite{debosscher2007automated,richards2011machine,dubath2011random}. The standard approach works as follows. A telescope observes a source $j$ at times $t_{1},\ldots,t_{l}$, recording flux measurements of $m_{1},\ldots,m_{l}$. Typically there are measurements of uncertainty on the flux measurements $e_{1},\ldots,e_{l}$. So each source $j$ is initially characterized by an $l \times 3$ matrix $D_j=\{(t_{i},m_{i},e_{i})\}_{i=1}^{l}$. Note that $l$, the number of times the source is observed, is different for each $D_j$. Also the time sampling of the flux measurements is irregular and different for each source. Associated with each source is a classification, such as Mira, Classical Cepheid, RR Lyrae, etc.

In order to construct a classifier, features are \textit{extracted} from $D_j$ (i.e. take functions of $D_j$) that will separate sources into different classes. Features vary from study to study, but typical ones include period (inverse of strongest fourier frequency), amplitude, skew, and estimates of derivatives.

If we compute $p$ features and have a total of $n$ training stars of known class ($D_1, \ldots D_n$), then we can use standard classification techniques on this $n\times p$ data matrix to construct a classifier. Given this classifier, we can then assign a class to a new source by extracting features and running the features through the classifier.


\subsection{Uncertain Features}
\begin{figure}[h]
  \begin{center}
    \begin{includegraphics}[scale=.5]{period_amplitude.pdf}
      \caption{Period-Amplitude relationship for two classes: \textit{Classical Cepheid} stars and \textit{Population II Cepheid} stars. The green box represents (roughly) the physically possible range of periods and amplitudes. Clearly some of these features have been estimated incorrectly.\label{fig:period_amplitude}}
    \end{includegraphics}
  \end{center}
\end{figure}
A major problem with the standard approach is the high levels and heteroskedastic nature of the uncertainty in the features due to having poorly sampled time series ($l$ is small), high error in the flux measurements ($e_{i}$ are systematically large), or irregular nature of sampling ($t_i$'s are highly concentrated, giving a poor representation of variation). This leads to situations where features meant to represent physical quantities of the time series, such as amplitude or period, are incorrect. As an example in Figure \ref{fig:period_amplitude} we plot the period and amplitude features for two classes of stars with a box around known physical limits for these features for these two classes. Some of the observations have values outside these physical limits, clearly indicating that the features are incorrectly estimated. This suggests that classifiers which incorporate feature uncertainty may be able to acheive improved performance.

%% Many of the features used for classification rely on a correct estimate of period for the object. For example, features related to the derivative are often computed based on the folded light curve. If the light curve is folded on the wrong period, then the derivative estimates are likely to be wrong.

\section{SVMs for Interval Data}
\subsection{Mathematical Formulation}

Support Vector Machines (SVMs) are a popular method for constructing classifiers \cite{scholkopf2002learning}. Letting $x_i \in \mathbb{R}^p$ be the vector of features for observation $i$ and $y_i$ its class (either $-1$ or $+1$), we can determine the SVM classifier by solving the optimization problem,
\begin{align}
\label{eq:dual}
& \min_{\beta,\beta_0}  \sum_{i=1}^n \xi_i + \frac{\lambda}{2}\beta^T\beta\\
& \text{subject to }  1 - y_if(x_i) \leq \xi_i; \, \xi \geq 0; \, f(x) = \beta_0 + \beta^Tx \nonumber
\end{align}
We determine the $\beta^*$ and $\beta_0^*$ that acheive the minimum and classify any new observation $x$ using $sign(\beta^* x + \beta_0)$. $\lambda$ is a tuning parameter than balances a tradeoff between maximizing the margin and separating observations in different classes. An optimal $\lambda$ may be chosen using 0-1 loss on a test set or through cross-validation.

Various extensions to this classifier have been proposed that seek to incorporate uncertainty in the features ($x$'s) into the optimization problem. These formulations often take (or can be interpreted) worst case 

- similarities to elastic net of LASSO / SVM (cite elastic net article)
- who has done L2 norm on location of features (

\subsection{Path Algorithm for Determining Optimal Tuning Parameter}

\section{Application to Variable Star Data Sets}
\subsection{Constructing the Intervals from the Time Series}
In certain applications, methods for constructing hyper-rectangles (or hyper-spheres) in feature space may be fairly straitforward. For example in El Ghaoui \cite{el2003robust}, the authors discuss applications to micro-array data where several replicates of some experiment are available, and a hyper-rectangle can be chosen to enclose all replicates.

With astronomy data there is no clearly correct way to construct intervals around features. A few possibilities include:
\begin{enumerate}
\item Make width of feature interval proportional to number of measurements in time series. Long time series will have small intervals around features, short time series will have wide intervals. For simple features, such as the standard deviation of the brightness measurements, making the interval width go down at rate root-n has a probabilistic interpretation via the central limit theorem.
\item Subsample the time series and derive features for each subsample. Represent observations as hyper-rectangles that contain features derived from every subsample (or a certain fraction of the subsamples).
\item Heuristically put intervals around observations with features that are wrong. For example, the amplitude of a star cannot be greater than 6 magnitudes, so any star with amplitude greater than 6 mags could be given an interval that contains values less than 6.
\end{enumerate}
In this work we take the second approach. While the first approach is attractive for simple features, many features (such as frequency) do not have a sqrt-n convergence to the true value. Further the irregular time sampling means that even simple features may not have a root-n convergence. The third approach involves a lot of domain knowledge and will change from application to application. Further, there is no guarantee that features within the range of what is physically possible are actually correct.

Figure \ref{fig:interval_construction} describes precisely how we subsample the time series and determine interval widths. The details of this procedure could be changed. For example one could bootstrap sample the time measurements, instead of subsampling contiguous sections. Certain computational considerations enter here. It takes around 1 second to derive features for an average length light curve. So sampling, say 20 times, could become prohibitively expansive if the data set is initially at the limits of what is computationally feasible.
\begin{figure*}[ht]
\begin{center}
{\small
\framebox[6in]{
\begin{minipage}[t]{5.8in}
% Insert here your text
\begin{center} \textbf{Constructing Hyper-rectangles from Time Series} \end{center}
\begin{enumerate}
\item Order the light curve measurements in time i.e. $\{(t_1,m_1,e_1),(t_2,m_2,e_2), \ldots , (t_l,m_l,e_l)\}$ where $t_i < t_j$ for $i < j$.
\item Slice each light curve into 5 contiguous sections, producing 5 time series. Slice 1 is $(t_i,m_i,e_i)$ for $\frac{0}{10}l \leq i \leq \frac{5}{10}l$, slice 2 is $\frac{1}{10}l \leq i \leq \frac{6}{10}l$, ect.
%% \begin{enumerate}
%% \item $\{(t_1,m_1,e_1), \ldots , (t_{l/2},m_{l/2},e_{l/2})\}$
%% \item $\{(t_1,m_1,e_1),(t_2,m_2,e_2), \ldots , (t_l,m_l,e_l)\}$
%% \end{enumerate}
\item Derive features for each of these slices.
\item Put the interval minimum at the lowest value of the feature for the 5 slices. Put the interval max at the maximum feature value across the 5 slices.
\end{enumerate}
\end{minipage}
}
}
\end{center}
\caption{Description of interval construction algorithm.\label{fig:interval_construction}}
\label{lb}
\end{figure*}

%% \begin{enumerate}
%% \item measurements in the middle of the light curve get used more often
%% \item sometimes the intervals do not contain the feature obtained when using the whole light curve. this is because taking shorter sections of light curves biases certain features high or low (frequency significance features tend to go down, should probably just get rid of these anyway)
%% \end{enumerate}
%% Bootstrap sampling might be a good approach, but this would destroy cadence sensitive features. So not sure what to do. This might be something interesting to discuss in the paper: \textbf{Determining feature error intervals for time series / image data is difficult.}




\section{Light Curves}

Figures \ref{fig:mira}, \ref{fig:cepheid}, \ref{fig:rrlyrae} show the light curves (i.e. time series) of three stars. Each is of a different class. The figures show the raw light curve (time on x-axis, magnitude on y-axis) and folded light curve (phase on x-axis, magnitude on y-axis). The phase is determined by computing a period (sometimes wrong!, for example figure \ref{fig:mira}) for each star, and then looking at the remainder of time divided by the period. The folded light curve often make the structure of the function much clearer.





\begin{figure}[h]
  \begin{center}
    \begin{includegraphics}[scale=.5]{4000.pdf}
      \caption{Light curve of a RR Lyrae AB variable star.\label{fig:rrlyrae}}
    \end{includegraphics}
  \end{center}
\end{figure}



\section{Initial Classification Work}

\subsection{Ignoring Errors}
I made a data set of 5508 of these light curves of class Mira, Classical Cepheid, and RR Lyrae AB. Random forest (a type of classifier) obtained a 0.55\% error rate, without using any feature error information. So the current classes may be too easy. I could generate features for some stars that belong to classes that are more similar to each other e.g. using stars from classes RR Lyrae C and RR Lyrae D (not used here) are a lot alike and would have much higher error rates.




\section{Notes on ``Robust Classification with Interval Data''}
\begin{enumerate}
\item how is new data point classified? this doesn't seem to be mentioned anywhere
\item role of $\rho$ is not super clear. described as a ``global measure of uncertainty'' on page 3 but then is then is optimized over e.g. equation 16
\item $\Sigma$ known and given
\item no mention of kernels for SVM with intervals. (kernels are mentioned with respect to MPM on page 11) perhaps SVM intervals don't work with intervals. that seems really bad. maybe if kernel preserves rectangular structure of interval?
\end{enumerate}

\bibliographystyle{plain}
\bibliography{refs}



\end{document}
